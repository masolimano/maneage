%% Copyright (C) 2018-2022 Mohammad Akhlaghi <mohammad@akhlaghi.org>
%% See the end of the file for license conditions.
\documentclass[10pt, twocolumn]{article}

%% (OPTIONAL) CONVENIENCE VARIABLE: Only relevant when you use Maneage's
%% '\includetikz' macro to build plots/figures within LaTeX using TikZ or
%% PGFPlots. If so, when the Figure files (PDFs) are already built, you can
%% avoid TikZ or PGFPlots completely by commenting/removing the definition
%% of '\makepdf' below. This is useful when you don't want to slow-down a
%% LaTeX-only build of the project (for example this happens when you run
%% './project make dist'). See the definition of '\includetikz' in
%% 'tex/preamble-pgfplots.tex' for more.
\newcommand{\makepdf}{}

%% VALUES FROM ANALYSIS (NUMBERS AND STRINGS): this file is automatically
%% generated at the end of the processing and includes LaTeX macros
%% (defined with '\newcommand') for various processing outputs to be used
%% within the paper.
\input{tex/build/macros/project.tex}

%% MANEAGE-ONLY PREAMBLE: this file contains LaTeX constructs that are
%% provided by Maneage (for example enabling or disabling of highlighting
%% from the './project' script). They are not style-related.
%% Preamble for Maneage-related features.
%
%% Copyright (C) 2020-2022 Mohammad Akhlaghi <mohammad@akhlaghi.org>
%
%% This LaTeX file is part of Maneage. Maneage is free software: you can
%% redistribute it and/or modify it under the terms of the GNU General
%% Public License as published by the Free Software Foundation, either
%% version 3 of the License, or (at your option) any later version.
%
%% Maneage is distributed in the hope that it will be useful, but WITHOUT
%% ANY WARRANTY; without even the implied warranty of MERCHANTABILITY or
%% FITNESS FOR A PARTICULAR PURPOSE.  See the GNU General Public License
%% for more details. See <http://www.gnu.org/licenses/>.





%% Highlighting
%% ------------
%
%% Maneage feature for highlighting changes they can be set activated
%% directly on the command-line with the '--highlight-new' or
%% '--highlight-notes' options
\ifdefined\highlightnew
\newcommand{\new}[1]{\textcolor{green!50!black}{#1}}
\else
\newcommand{\new}[1]{\textcolor{black}{#1}}
\fi

\ifdefined\highlightnotes
\newcommand{\tonote}[1]{\textcolor{red!60!black}{[#1]}}
\else
\newcommand{\tonote}[1]{{}}
\fi


%% PROJECT-SPECIFIC PREAMBLE: This is where you can include any LaTeX
%% setting for customizing your project.
%% Necessary macros for this project.
%%
%% These are a set of packages that have been commonly necessary in most
%% LaTeX usages. However, if any are not needed in your work, please feel
%% free to remove them.
%
%% Copyright (C) 2018-2022 Mohammad Akhlaghi <mohammad@akhlaghi.org>
%% Copyright (C) YYYY Your Name <your@email.address>
%
%% This file is free software: you can redistribute it and/or modify it
%% under the terms of the GNU General Public License as published by the
%% Free Software Foundation, either version 3 of the License, or (at your
%% option) any later version.
%
%% This file is distributed in the hope that it will be useful, but WITHOUT
%% ANY WARRANTY; without even the implied warranty of MERCHANTABILITY or
%% FITNESS FOR A PARTICULAR PURPOSE.  See the GNU General Public License
%% for more details.
%
%% You should have received a copy of the GNU General Public License along
%% with this file.  If not, see <http://www.gnu.org/licenses/>.





%% Packages you may need in your project
%% -------------------------------------
%
%% Here you can add/remove any custom LaTeX package that you need for this
%% project that aren't provided by the journal's style.

%% For loading images into the output (with '\includegraphics').
\usepackage{graphicx}

%% Ordering correction between 'figure' and 'figure*' ('figure*' is
%% commonly used in two-column documents, where the figure should span both
%% columns).
\usepackage{fixltx2e}

%% Color management.
\usepackage{xcolor}
\color{black} % Color of main text.
\definecolor{DarkBlue}{RGB}{0,0,90}

%% Caption management: The 'setspace' package defines the 'stretch'
%% variable. 'abovecaptionskip' is the distance between the figure and the
%% caption. You can use 'captionof{figure}{...}' to use these custom
%% 'figure' caption that is defined here.
\usepackage{setspace, caption}
\captionsetup{font=footnotesize, labelfont={color=DarkBlue,bf}, skip=1pt}
\captionsetup[figure]{font={stretch=1, small}}
\setlength{\abovecaptionskip}{3pt plus 1pt minus 1pt}
\setlength{\belowcaptionskip}{-1.25em}

%% Manage links in the produced paper (for example their colors), and
%% include document information in the "Properties" of the PDF.
\usepackage[
  colorlinks,
  urlcolor=blue,
  citecolor=blue,
  linkcolor=blue,
  linktocpage]{hyperref}
\renewcommand\UrlFont{\rmfamily}
\hypersetup{
    pdftitle={\projecttitle},
    pdfauthor={\projectcopyrightowner},
    pdfsubject={\projectgitrepo{} (commit \projectversion)},
    pdfkeywords={Reproducible research, Maneage, ADD YOUR OWN}
}





%% BibLaTeX or PGFPlots templates
%% ------------------------------
%
%% These are ready-made customizations of these two commonly used packages
%% that you can use as a template for your own project: BibLaTeX (advanced
%% bibliography management) or PGFPlots (for drawing plots within LaTeX
%% directly from tables of data). If you don't use them, you can just
%% delete these two lines and also delete their files from your branch (to
%% keep the 'tex/src' directory on your branch clean).
%% Biblatex settings.
%%
%% Settings necessary to make the bibliography with Biblatex. Keeping all
%% BibLaTeX settings in a separate preamble was done in the spirit of
%% modularity to 1) easily managable, 2) If a similar BibLaTeX
%% configuration is necessary in another LaTeX compilation, this file can
%% just be copied there and used.
%%
%% USAGE:
%%  - 'tex/src/references.tex': the file containing Bibtex source of each
%%     reference. The file suffix doesn't have to be '.bib'. This naming
%%     helps in clearly identifying the files and avoiding places that
%%     complain about '.bib' files.
%
%% Copyright (C) 2018-2022 Mohammad Akhlaghi <mohammad@akhlaghi.org>
%
%% This file is free software: you can redistribute it and/or modify it
%% under the terms of the GNU General Public License as published by the
%% Free Software Foundation, either version 3 of the License, or (at your
%% option) any later version.
%
%% This file is distributed in the hope that it will be useful, but WITHOUT
%% ANY WARRANTY; without even the implied warranty of MERCHANTABILITY or
%% FITNESS FOR A PARTICULAR PURPOSE.  See the GNU General Public License
%% for more details.
%
%% You should have received a copy of the GNU General Public License along
%% with this file.  If not, see <http://www.gnu.org/licenses/>.




%% To break up highlighted text (for example texttt when some it is on the
%% line break) and also to no underline emphasized words (like journal
%% titles in the references).
\usepackage[normalem]{ulem}





%% For quotation signs (sometimes used by BibLaTeX)
\usepackage{csquotes}





%% To define colors
\usepackage{xcolor}





% Basic BibLaTeX settings
\usepackage[
    doi=false,
    url=false,
    dashed=false,
    eprint=false,
    maxbibnames=2,
    minbibnames=1,
    hyperref=true,
    maxcitenames=2,
    mincitenames=1,
    giveninits=true,
    style=authoryear,
    uniquelist=false,
    backend=biber,natbib]{biblatex}
\DeclareFieldFormat[article]{pages}{#1}
\DeclareFieldFormat{pages}{\mkfirstpage[{\mkpageprefix[bookpagination]}]{#1}}
\addbibresource{tex/src/references.tex}
\addbibresource{tex/build/macros/dependencies-bib.tex}
\renewbibmacro{in:}{}
\AtEveryBibitem{\clearfield{month}}
\renewcommand*{\bibfont}{\footnotesize}
\DefineBibliographyStrings{english}{references = {References}}

%% Include the adsurl field key into those that are recognized:
\DeclareSourcemap{
  \maps[datatype=bibtex]{
    \map{
      \step[fieldsource=adsurl,fieldtarget=iswc]
      \step[fieldsource=gbkurl,fieldtarget=iswc]
    }
  }
}

%% Set the color of the doi link to mymg (magenta) and the ads links
%% to mypurp (or purple):
\definecolor{mypurp}{cmyk}{0.75,1,0,0}
\newcommand{\doihref}[2]{\href{#1}{\color{magenta}{#2}}}
\newcommand{\adshref}[2]{\href{#1}{\color{mypurp}{#2}}}
\newcommand{\blackhref}[2]{\href{#1}{\color{black}{#2}}}

%% Define a format for the printtext commands in
%% DeclareBibliographyDriver to make links for the doi, ads link and
%% arxiv link:
\DeclareFieldFormat{doilink}{
  \iffieldundef{doi}{#1}{\doihref{http://dx.doi.org/\thefield{doi}}{#1}}}
\DeclareFieldFormat{adslink}{
    \iffieldundef{iswc}{#1}{\adshref{\thefield{iswc}}{#1}}}
\DeclareFieldFormat{arxivlink}{
  \iffieldundef{eprint}{#1}{\href{http://arxiv.org/abs/\thefield{eprint}}{#1}}}

\DeclareListFormat{doiforbook}{
  \iffieldundef{doi}{#1}{\doihref{http://dx.doi.org/\thefield{doi}}{#1}}}
\DeclareFieldFormat{googlebookslink}{
    \iffieldundef{iswc}{#1}{\adshref{\thefield{iswc}}{#1}}}

%% Set the formatting to make the last three values into the
%% appropriate link. Note that the % signs are necessary. Without
%% them, the items will be indented.
\DeclareBibliographyDriver{article}{%
  \usebibmacro{bibindex}%
  \usebibmacro{begentry}%
  \usebibmacro{author/translator+others}%
  \newunit%
  \ifdefined\makethesis\printtext{\usebibmacro{title}}\fi%
  \newunit%
  \printtext[doilink]{\usebibmacro{journal}}%
  \addcomma%
  \printtext[adslink]{\printfield{volume}}%
  \addcomma%
  \printtext[arxivlink]{\printfield{pages}}%
  \addperiod%
}

\DeclareBibliographyDriver{book}{%
  \usebibmacro{bibindex}%
  \usebibmacro{begentry}%
  \usebibmacro{author/translator+others}%
  \newunit%
  \printtext{\usebibmacro{title}}%
  \addperiod%
  \addspace%
  \printlist[doiforbook]{publisher}%
  \addcomma%
  \addspace%
  \printfield[googlebookslink]{edition}%
  \printtext{ ed.}%
  \addperiod%
}

%% In order to have et al. instead of et al.,:
\renewcommand*{\nameyeardelim}{\addspace}

%% PGFPlots settings
%% -----------------
%
%% PGFPLOTS is a package in (La)TeX for making plots internally. It fits
%% nicely with the purpose of a reproducible project. But it isn't
%% mandatory. Therefore if you don't need it, just comment/delete the line
%% that includes this file in the top LaTeX source ('paper.tex').
%
%% However, TiKZ and PGFPlots are the recommended way to include figures
%% and plots in your paper. There are two main reasons: 1) it follows the
%% same LaTeX settings as the text of the paper, so the figures will be in
%% the exact same settings (for example font or lines) as the main body of
%% the papers. 2) It doesn't require any extra dependency (it is
%% distributed as part of TeX-live). Adding specific programs/libraries for
%% plots can greatly increase the number of dependencies for the
%% project. For example Python's Matplotlib library is indeed very good,
%% but it requires Python and Numpy. The latter is not easy to build from
%% source, so after a few years, installing the required version can be
%% very frustrating.
%
%% Keeping all BibLaTeX settings in a separate preamble was done in the
%% spirit of modularity to 1) easily managable, 2) If a similar BibLaTeX
%% configuration is necessary in another LaTeX compilation, this file can
%% just be copied there and used.
%
%% PGFPlots uses the (La)TeX TiKZ package to build plots. So we will first
%% do the settings that are necessary in TiKZ, and then go onto the actual
%% PGFPlots package.
%%
%% USAGE:
%
%%  - All plots are made within a 'tikz' directory (that must already be
%%    present in the location LaTeX is run).
%
%%  - Use '\includetikz{XXXX}' to make/use the figure. If a 'makepdf' LaTeX
%%    macro is not defined, then \includetikz will assume a 'XXXX.pdf' file
%%    exists in 'tex/tikz' and simply import it. If 'makepdf' is defined,
%%    then TiKZ/PGFPlot will be called to (possibly) build the plot based
%%    on 'tex/XXXX.tex'. Note that if the contents of 'tex/src/XXXX.tex'
%%    hasn't changed since the last build. TiKZ/PGFPlots won't rebuild the
%%    plot.
%
%% Copyright (C) 2018-2022 Mohammad Akhlaghi <mohammad@akhlaghi.org>
%
%% This file is part of Maneage (https://maneage.org).
%
%% This file is free software: you can redistribute it and/or modify it
%% under the terms of the GNU General Public License as published by the
%% Free Software Foundation, either version 3 of the License, or (at your
%% option) any later version.
%
%% This file is distributed in the hope that it will be useful, but WITHOUT
%% ANY WARRANTY; without even the implied warranty of MERCHANTABILITY or
%% FITNESS FOR A PARTICULAR PURPOSE.  See the GNU General Public License
%% for more details.
%
%% You should have received a copy of the GNU General Public License along
%% with this file.  If not, see <http://www.gnu.org/licenses/>.




%% Very general TiKZ settings. In particular, to allow faster processing
%% (not having to re-build the plots on every run), we are using the
%% externalization feature of TiKZ. With this option, TiKZ will build every
%% figure independently in a special directory afterwards it will include
%% the built figure in the final file. This has many advantages: 1) if the
%% code for the plot hasn't changed, then the plot won't be re-made (can be
%% slow with detailed plots). 2) You can use the PDFs of the individual
%% plots for other purposes (for example to include in slides) cleanly.
\usepackage{tikz}
\usetikzlibrary{external}
\tikzexternalize
\tikzsetexternalprefix{tikz/}





%% The '\includetikz' can be used to either build the figures using
%% PGFPlots (when '\makepdf' is defined), or use an existing file (when
%% '\makepdf' isn't defined). When making the PDF, it will set the output
%% figure name to be the same as the 'tex/src/XXXX.tex' file that contains
%% the PGFPlots source of the figure. In this way, when using the PDF, it
%% will also have the same name, thus allowing the figures to easily change
%% their place relative to others: figure ordering won't be a problem. This
%% is a problem by default because if an explicit name isn't set at the
%% start, tikz will make images based on their order in the paper.
%
%% This function takes two arguments:
%%     1) The base-name of the LaTeX file with the 'tikzpicture'
%%        environment. As mentioned above, this will also be the name of
%%        the produced figure.
%%     2) The settings to use with 'includegraphics' when an already-built
%%        file should be used.
\newcommand{\includetikz}[2]{%
  \ifdefined\makepdf%
    \tikzsetnextfilename{#1}%
    \input{tex/src/#1.tex}%
  \else
    \includegraphics[#2]{tex/tikz/#1.pdf}
  \fi
}





%% Uncomment the following lines for EPS and PS images. Note that you still
%% have to use the 'pdflatex' executable and also add a '[dvips]' option to
%% graphicx.

%% \tikzset{external/system call={rm -f "\image".eps "\image".ps
%% "\image".dvi; latex \tikzexternalcheckshellescape -halt-on-error
%% -interaction=batchmode -jobname "\image" "\texsource";
%% dvips -o "\image".ps "\image".dvi;
%% ps2eps "\image.ps"}}





%% Inport and configure PGFPlots.
\usepackage{pgfplots}
\pgfplotsset{compat=newest}
\usepgfplotslibrary{groupplots}
\pgfplotsset{
  axis line style={thick},
  tick style={semithick},
  tick label style = {font=\footnotesize},
  every axis label = {font=\footnotesize},
  legend style = {font=\footnotesize},
  label style = {font=\footnotesize}
  }






%% Style of default paper (DELETE IF USING JOURNAL STYLES)
%% -------------------------------------------------------
%
%% This is primarily defined for the default Maneage paper style. So when
%% you later import your journal's style, delete this line (and these
%% comments). Also delete the file (to keep your project source branch
%% clean from files you don't need/use).
%% General paper's style settings.
%
%% This preamble can be completely ignored when including this TeX file in
%% another style. This is done because this LaTeX build is meant to be an
%% initial/internal phase or part of a larger effort, so it has a basic
%% style defined here as a preamble. To ignore it, uncomment or delete the
%% respective line in 'paper.tex'.
%
%% Copyright (C) 2019-2022 Mohammad Akhlaghi <mohammad@akhlaghi.org>
%
%% This file is free software: you can redistribute it and/or modify it
%% under the terms of the GNU General Public License as published by the
%% Free Software Foundation, either version 3 of the License, or (at your
%% option) any later version.
%
%% This file is distributed in the hope that it will be useful, but WITHOUT
%% ANY WARRANTY; without even the implied warranty of MERCHANTABILITY or
%% FITNESS FOR A PARTICULAR PURPOSE.  See the GNU General Public License
%% for more details.
%
%% You should have received a copy of the GNU General Public License along
%% with this file.  If not, see <http://www.gnu.org/licenses/>.





%% Font.
\usepackage[T1]{fontenc}
\usepackage{newtxtext}
\usepackage{newtxmath}





%% Print size
\usepackage[a4paper, includeheadfoot, body={18.7cm, 24.5cm}]{geometry}





%% Set the distance between the columns if two columns:
\setlength{\columnsep}{0.75cm}





% To allow figures to take up more space on the top of the page:
\renewcommand{\topfraction}{.99}
\renewcommand{\bottomfraction}{.7}
\renewcommand{\textfraction}{.05}
\renewcommand{\floatpagefraction}{.99}
\renewcommand{\dbltopfraction}{.99}
\renewcommand{\dblfloatpagefraction}{.99}
\setcounter{topnumber}{1}
\setcounter{bottomnumber}{0}
\setcounter{totalnumber}{2}
\setcounter{dbltopnumber}{1}





%% To make the footnotes align:
\usepackage[hang]{footmisc}
\setlength\footnotemargin{10pt}





%For including time in the title:
\usepackage{datetime}





%% Define the abstract environment
\renewenvironment{abstract}
 {\vspace{-0.5cm}\small%
  \list{}{%
    \setlength{\leftmargin}{2cm}%
    \setlength{\rightmargin}{\leftmargin}%
  }%
  \item\relax}
 {\endlist}





%% To keep the main page's code clean.
\newcommand{\includeabstract}[1]{%
\twocolumn[%
  \begin{@twocolumnfalse}%
    \maketitle%
    \begin{abstract}%
    #1%
    \end{abstract}%
    \vspace{1cm}%
  \end{@twocolumnfalse}%
  ]%
}





%% Basic header style
%% ------------------
%
%% The steps below are to use the necessary LaTeX packages to get the demo
%% Maneage paper running with a reasonably looking, custom paper style. If
%% you are using a custom journal style, feel free to delete these.

%% General page header settings.
\usepackage{fancyhdr}
\pagestyle{fancy}
\lhead{\footnotesize{\scshape Draft paper}, {\footnotesize nnn:i (pp), Year Month day}}
\rhead{\scshape\footnotesize YOUR-NAME et al.}
\cfoot{\thepage}
\setlength{\voffset}{0.75cm}
\setlength{\headsep}{0.2cm}
\setlength{\footskip}{0.75cm}
\renewcommand{\headrulewidth}{0pt}

%% Specific style for first page.
\fancypagestyle{firststyle}
{
  \lhead{\footnotesize{\scshape Draft paper}, nnn:i (pp), YYYY Month day\\
  \scriptsize \textcopyright YYYY, Your name. All rights reserved.}
  \rhead{\footnotesize \footnotesize \today, \currenttime\\}
}

%To set the style of the titles:
\usepackage{titlesec}
\titleformat{\section}
  {\centering\normalfont\uppercase}
  {\thesection.}
  {0em}
  { }
\titleformat{\subsection}
  {\centering\normalsize\slshape}
  {\thesubsection.}
  {0em}
  { }
\titleformat{\subsubsection}
  {\centering\small\slshape}
  {\thesubsubsection.}
  {0em}
  { }

%% Title and author information
\usepackage{authblk}
\renewcommand\Authfont{\small\scshape}
\renewcommand\Affilfont{\footnotesize\normalfont}
\setlength{\affilsep}{0.2cm}







%% PROJECT TITLE: The project title should also be printed as metadata in
%% all output files. To avoid inconsistancy caused by manually typing it,
%% the title is defined with other core project metadata in
%% 'reproduce/analysis/config/metadata.conf'. That value is then written in
%% the '\projectitle' LaTeX macro which is available in 'project.tex' (that
%% was loaded above).
%
%% Please set your project's title in 'metadata.conf' (ideally with other
%% basic project information) and re-run the project to have your new
%% title. If you later use a different LaTeX style, please use the same
%% '\projectitle' in it (after importing 'tex/build/macros/project.tex'
%% like above), don't type it by hand.
\title{\large \uppercase{\projecttitle}}

%% AUTHOR INFORMATION: For a more fine-grained control of the headers
%% including author name, or paper info, see
%% 'tex/src/preamble-header.tex'. Note that if you plan to use a journal's
%% LaTeX style file, you will probably set the authors in a different way,
%% feel free to change them here, this is just basic style and varies from
%% project to project.
\author[1]{Your name}
\author[2]{Coauthor one}
\author[1,3]{Coauthor two}
\affil[1]{The first affiliation in the list.; \url{your@email.address}}
\affil[2]{Another affilation can be put here.}
\affil[3]{And generally as many affiliations as you like.
\par \emph{Received YYYY MM DD; accepted YYYY MM DD; published YYYY MM DD}}
\date{}










%% Start creating the paper.
\begin{document}

%% Project abstract and keywords.
\includeabstract{
  Welcome to Maneage (\emph{Man}aging data lin\emph{eage}) and reproducible papers/projects, for a review of the basics of this system, please see \citet{maneage}.
  You are now ready to configure Maneage and implement your own research in this framework.
  Maneage contains almost all the elements that you will need in a research project, and adding any missing parts is very easy once you become familiar with it.
  For example it already has steps to downloading of raw data and necessary software (while verifying them with their checksums), building the software, and processing the data with the software in a highly-controlled environment.
  But Maneage is not just for the analysis of your project, you will also write your paper in it (by replacing this text in \texttt{paper.tex}): including this abstract, figures and bibliography.
  If you design your project with Maneage's infra-structure, don't forget to add a notice and clearly let the readers know that your work is reproducible, we should spread the word and show the world how useful reproducible research is for the sciences, also don't forget to cite and acknowledge it so we can continue developing it.
  This PDF was made with Maneage, commit \projectversion{}.

  \vspace{0.25cm}

  \textsl{Keywords}: Add some keywords for your research here.

  \textsl{Reproducible paper}: All quantitave results (numbers and plots)
  in this paper are exactly reproducible with Maneage
  (\url{https://maneage.org}).  }

%% To add the first page's headers.
\thispagestyle{firststyle}





%% Start of main body.
\section{Congratulations!}
Congratulations on running the raw template project! You can now follow the ``Customization checklist'' in the \texttt{README-hacking.md} file, customize this template and start your exciting research project over it.
You can always merge Maneage back into your project to improve its infra-structure and leaving your own project intact.
If you haven't already read \citet{maneage}, please do so before continuing, it isn't long (just 7 pages).

While you are writing your paper, just don't forget to \emph{not} use numbers or fixed strings (for example database urls like \url{\wfpctwourl}) directly within your \LaTeX{} source.
Put them in configuration files and after using them in the analysis, pass them into the \LaTeX{} source through macros in the same subMakefile that used them.
For some already published examples, please see \citet{maneage}\footnote{\url{https://gitlab.com/makhlaghi/maneage-paper}}, \citet{infantesainz20}\footnote{\url{https://gitlab.com/infantesainz/sdss-extended-psfs-paper}} and \citet{akhlaghi19}\footnote{\url{https://gitlab.com/makhlaghi/iau-symposium-355}}.
Working in this way, will let you focus clearly on your science and not have to worry about fixing this or that number/name in the text.

Once your project is ready for publication, there is also a ``Publication checklist'' in \texttt{README-hacking.md} that will guide you in the steps to do for making your project as FAIR as possible (Findable, Accessibile, Interoperable, and Reusable).

The default \LaTeX{} structure within Maneage also has two \LaTeX{} macros for easy marking of text within your document as \emph{new} and \emph{notes}.
To activate them, please use the \texttt{--highlight-new} or \texttt{--highlight-notes} options with \texttt{./project make}.

For example if you run \texttt{./project make --highlight-new}, then \new{this text (that has been marked as \texttt{new}) will show up as green in the final PDF}.
If you run \texttt{./project make --highlight-notes} then you will see a note following this sentence that is written in red and has square brackets around it (it is invisible without this option).
\tonote{This text is written within a \texttt{tonote} and is invisible without \texttt{--highlight-notes}.}
You can also use these two options together to both highlight the new parts and add notes within the text.

Another thing you may notice from the \LaTeX{} source of this default paper is there is one line per sentence (and one sentence in a line).
Of course, as with everything else in Maneage, you are free to use any format that you are most comfortable with.
The reason behind this choice is that this source is under Git version control and that Git also primarily works on lines.
In this way, when a change in a setence is made, git will only highlight/color that line/sentence we have found that this helps a lot in viewing the changes.
Also, this format helps in reminding the author when the sentence is getting too long!
Here is a tip when looking at the changes of narrative document in Git: use the \texttt{--word-diff} option (for example \texttt{git diff --word-diff}, you can also use it with \texttt{git log}).

Figure \ref{squared} shows a simple plot as a demonstration of creating plots within \LaTeX{} (using the {\small PGFP}lots package).
The minimum value in this distribution is $\deletememin$, and $\deletememax$ is the maximum.
Take a look into the \LaTeX{} source and you'll see these numbers are actually macros that were calculated from the same dataset (they will change if the dataset, or function that produced it, changes).

The individual {\small PDF} file of Figure \ref{squared} is available under the \texttt{tex/tikz/} directory of your build directory.
You can use this PDF file in other contexts (for example in slides showing your progress or after publishing the work).
If you want to directly use the {\small PDF} file in the figure without having to let {\small T}i{\small KZ} decide if it should be remade or not, you can also comment the \texttt{makepdf} macro at the top of this \LaTeX{} source file.

\begin{figure}[t]
  \includetikz{delete-me-squared}{width=\linewidth}

  \captionof{figure}{\label{squared} A very basic $X^2$ plot for
    demonstration.}
\end{figure}

Figure \ref{image-histogram} is another demonstration of showing images (datasets) using PGFPlots.
It shows a small crop of an image from the Wide-Field Planetary Camera 2 (that was installed on the Hubble Space Telescope from 1993 to 2009).
As another more realistic demonstration of reporting results with Maneage, here we report that the mean pixel value in that image is $\deletemewfpctwomean$ and the median is $\deletemewfpctwomedian$.
The skewness in the histogram of Figure \ref{image-histogram}(b) explains this difference between the mean and median.
The dataset is visualized here as a black and white image using the \textsf{Convert\-Type} program of GNU Astronomy Utilities (Gnuastro).
The histogram and basic statstics were generated with Gnuastro's \textsf{Statistics} program.

{\small PGFP}lots\footnote{\url{https://ctan.org/pkg/pgfplots}} is a great tool to build the plots within \LaTeX{} and removes the necessity to add further dependencies, just to create the plots.
There are high-level libraries like Matplotlib which also generate plots.
However, the problem is that they require \emph{many} dependencies, for example see Figure 1 of \citet{alliez19}.
Installing these dependencies from source, is not easy and will harm the reproducibility of your paper in the future.

Furthermore, since {\small PGFP}lots builds the plots within \LaTeX{}, it respects all the properties of your text (for example line width and fonts and etc).
Therefore the final plot blends in your paper much more nicely.
It also has a wonderful manual\footnote{\url{http://mirrors.ctan.org/graphics/pgf/contrib/pgfplots/doc/pgfplots.pdf}}.

\begin{figure}[t]
  \includetikz{delete-me-image-histogram}{width=\linewidth}

  \captionof{figure}{\label{image-histogram} (a) An example image of the Wide-Field Planetary Camera 2, on board the Hubble Space Telescope from 1993 to 2009.
    This is one of the sample images from the FITS standard webpage, kept as examples for this file format.
    (b) Histogram of pixel values in (a).}
\end{figure}



\section{Notice and citations}
To encourage other scientists to publish similarly reproducible papers, please add a notice close to the start of your paper or in the end of the abstract clearly mentioning that your work is fully reproducible.
One convention we have adopted until now is to put the Git checkum of the project as the last word of the abstract, for example see \citet{akhlaghi19}, \citet{infantesainz20} and \citet{maneage}

Finally, don't forget to cite \citet{maneage} and acknowledge the funders mentioned below.
Otherwise we won't be able to continue working on Maneage.
Also, just as another reminder, before publication, don't forget to follow the ``Publication checklist'' of \texttt{README-hacking.md}.
%% End of main body.





\section{Acknowledgments}
\new{Please include the following paragraph in the Acknowledgement section of your paper.
  In order to get more funding to continue working on Maneage, we need to to cite it and its funding institutions in your papers.
  Also note that at the start, it includes version and date information for the most recent Maneage commit you merged with (which can be very helpful for others) as well as very basic information about your CPU architecture (which was extracted during configuration).
  This CPU information is very important for reproducibility because some software may not be buildable on other CPU architectures, so it is necessary to publish CPU information with the results and software versions.}

This project was developed in the reproducible framework of Maneage \citep[\emph{Man}aging data lin\emph{eage},][latest Maneage commit \maneageversion{}, from \maneagedate]{maneage}.
The project was built on an {\machinearchitecture} machine with {\machinebyteorder} byte-order, see Appendix \ref{appendix:software} for the used software and their versions.
Maneage has been funded partially by the following grants: Japanese Ministry of Education, Culture, Sports, Science, and Technology (MEXT) PhD scholarship to M. Akhlaghi and its Grant-in-Aid for Scientific Research (21244012, 24253003).
The European Research Council (ERC) advanced grant 339659-MUSICOS.
The European Union (EU) Horizon 2020 (H2020) research and innovation programmes No 777388 under RDA EU 4.0 project, and Marie Sk\l{}odowska-Curie grant agreement No 721463 to the SUNDIAL ITN.
The State Research Agency (AEI) of the Spanish Ministry of Science, Innovation and Universities (MCIU) and the European Regional Development Fund (ERDF) under the grant AYA2016-76219-P.
The IAC project P/300724, financed by the MCIU, through the Canary Islands Department of Economy, Knowledge and Employment.

%% Tell BibLaTeX to put the bibliography list here.
\printbibliography

%% Start appendix.
\appendix

%% Mention all used software in an appendix.
\section{Software acknowledgement}
\label{appendix:software}
\input{tex/build/macros/dependencies.tex}

%% Finish LaTeX
\end{document}

%% This file is part of Maneage (https://maneage.org).
%
%% This file is free software: you can redistribute it and/or modify it
%% under the terms of the GNU General Public License as published by the
%% Free Software Foundation, either version 3 of the License, or (at your
%% option) any later version.
%
%% This file is distributed in the hope that it will be useful, but WITHOUT
%% ANY WARRANTY; without even the implied warranty of MERCHANTABILITY or
%% FITNESS FOR A PARTICULAR PURPOSE.  See the GNU General Public License
%% for more details.
%
%% You should have received a copy of the GNU General Public License along
%% with this file.  If not, see <http://www.gnu.org/licenses/>.
