%% Some (commonly) necessary LaTeX packages.
%%
%% These are a set of packages that have been commonly necessary in most
%% LaTeX usages. However, if any are not needed in your work, please feel
%% free to remove them.





% Macros for to help in typing, remove them if you don't need them, but
% this can help as a demo on how you can simply writing of commonly used
% words that need special formatting (like software names).
\newcommand{\gnu}[1]{{\small GNU} #1}
\newcommand{\snsign}{{\small S}/{\small N}}
\newcommand{\originsoft}{\textsf{ORIGIN}}
\newcommand{\sextractor}{\textsf{SE\-xtractor}}
\newcommand{\noisechisel}{\textsf{Noise\-Chisel}}
\newcommand{\makecatalog}{\textsf{Make\-Catalog}}





%% For highlighting updates. When this is set, text marked as \new
%% will be colored in dark green and text that is marked wtih \tonote
%% will be marked in dark red.
\ifdefined\highlightchanges
\newcommand{\new}[1]{\textcolor{green!60!black}{#1}}
\newcommand{\tonote}[1]{\textcolor{red!60!black}{[#1]}}
\else
\newcommand{\new}[1]{\textcolor{black}{#1}}
\newcommand{\tonote}[1]{{}}
\fi





% Better than verbatim for displaying typed text.
\usepackage{alltt}





% For arithmetic opertions within LaTeX
\usepackage[nomessages]{fp}





%To add a code font to the text:
\usepackage{courier}





%To add some enumerating styles
\usepackage{enumerate}





%Including images if necessary
\usepackage{graphicx}
