%% Necessary macros for this project.
%%
%% These are a set of packages that have been commonly necessary in most
%% LaTeX usages. However, if any are not needed in your work, please feel
%% free to remove them.
%
%% Copyright (C) 2018-2021 Mohammad Akhlaghi <mohammad@akhlaghi.org>
%% Copyright (C) YYYY Your Name <your@email.address>
%
%% This file is free software: you can redistribute it and/or modify it
%% under the terms of the GNU General Public License as published by the
%% Free Software Foundation, either version 3 of the License, or (at your
%% option) any later version.
%
%% This file is distributed in the hope that it will be useful, but WITHOUT
%% ANY WARRANTY; without even the implied warranty of MERCHANTABILITY or
%% FITNESS FOR A PARTICULAR PURPOSE.  See the GNU General Public License
%% for more details.
%
%% You should have received a copy of the GNU General Public License along
%% with this file.  If not, see <http://www.gnu.org/licenses/>.





%% Packages you may need in your project
%% -------------------------------------
%
%% Here you can add/remove any custom LaTeX package that you need for this
%% project that aren't provided by the journal's style.

%% For loading images into the output (with '\includegraphics').
\usepackage{graphicx}

%% Ordering correction between 'figure' and 'figure*' ('figure*' is
%% commonly used in two-column documents, where the figure should span both
%% columns).
\usepackage{fixltx2e}

%% Color management.
\usepackage{xcolor}
\color{black} % Color of main text.
\definecolor{DarkBlue}{RGB}{0,0,90}

%% Caption management: The `setspace' package defines the `stretch'
%% variable. `abovecaptionskip' is the distance between the figure and the
%% caption. You can use 'captionof{figure}{...}' to use these custom
%% 'figure' caption that is defined here.
\usepackage{setspace, caption}
\captionsetup{font=footnotesize, labelfont={color=DarkBlue,bf}, skip=1pt}
\captionsetup[figure]{font={stretch=1, small}}
\setlength{\abovecaptionskip}{3pt plus 1pt minus 1pt}
\setlength{\belowcaptionskip}{-1.25em}

%% Manage links in the produced paper (for example their colors), and
%% include document information in the "Properties" of the PDF.
\usepackage[
  colorlinks,
  urlcolor=blue,
  citecolor=blue,
  linkcolor=blue,
  linktocpage]{hyperref}
\renewcommand\UrlFont{\rmfamily}
\hypersetup{
    pdftitle={\projecttitle},
    pdfauthor={\projectcopyrightowner},
    pdfsubject={\projectgitrepo{} (commit \projectversion)},
    pdfkeywords={Reproducible research, Maneage, ADD YOUR OWN}
}





%% BibLaTeX or PGFPlots templates
%% ------------------------------
%
%% These are ready-made customizations of these two commonly used packages
%% that you can use as a template for your own project: BibLaTeX (advanced
%% bibliography management) or PGFPlots (for drawing plots within LaTeX
%% directly from tables of data). If you don't use them, you can just
%% delete these two lines and also delete their files from your branch (to
%% keep the 'tex/src' directory on your branch clean).
%% Biblatex settings.
%%
%% Settings necessary to make the bibliography with Biblatex. Keeping all
%% BibLaTeX settings in a separate preamble was done in the spirit of
%% modularity to 1) easily managable, 2) If a similar BibLaTeX
%% configuration is necessary in another LaTeX compilation, this file can
%% just be copied there and used.
%%
%% USAGE:
%%  - 'tex/src/references.tex': the file containing Bibtex source of each
%%     reference. The file suffix doesn't have to be '.bib'. This naming
%%     helps in clearly identifying the files and avoiding places that
%%     complain about '.bib' files.
%
%% Copyright (C) 2018-2022 Mohammad Akhlaghi <mohammad@akhlaghi.org>
%
%% This file is free software: you can redistribute it and/or modify it
%% under the terms of the GNU General Public License as published by the
%% Free Software Foundation, either version 3 of the License, or (at your
%% option) any later version.
%
%% This file is distributed in the hope that it will be useful, but WITHOUT
%% ANY WARRANTY; without even the implied warranty of MERCHANTABILITY or
%% FITNESS FOR A PARTICULAR PURPOSE.  See the GNU General Public License
%% for more details.
%
%% You should have received a copy of the GNU General Public License along
%% with this file.  If not, see <http://www.gnu.org/licenses/>.




%% To break up highlighted text (for example texttt when some it is on the
%% line break) and also to no underline emphasized words (like journal
%% titles in the references).
\usepackage[normalem]{ulem}





%% For quotation signs (sometimes used by BibLaTeX)
\usepackage{csquotes}





%% To define colors
\usepackage{xcolor}





% Basic BibLaTeX settings
\usepackage[
    doi=false,
    url=false,
    dashed=false,
    eprint=false,
    maxbibnames=2,
    minbibnames=1,
    hyperref=true,
    maxcitenames=2,
    mincitenames=1,
    giveninits=true,
    style=authoryear,
    uniquelist=false,
    backend=biber,natbib]{biblatex}
\DeclareFieldFormat[article]{pages}{#1}
\DeclareFieldFormat{pages}{\mkfirstpage[{\mkpageprefix[bookpagination]}]{#1}}
\addbibresource{tex/src/references.tex}
\addbibresource{tex/build/macros/dependencies-bib.tex}
\renewbibmacro{in:}{}
\AtEveryBibitem{\clearfield{month}}
\renewcommand*{\bibfont}{\footnotesize}
\DefineBibliographyStrings{english}{references = {References}}

%% Include the adsurl field key into those that are recognized:
\DeclareSourcemap{
  \maps[datatype=bibtex]{
    \map{
      \step[fieldsource=adsurl,fieldtarget=iswc]
      \step[fieldsource=gbkurl,fieldtarget=iswc]
    }
  }
}

%% Set the color of the doi link to mymg (magenta) and the ads links
%% to mypurp (or purple):
\definecolor{mypurp}{cmyk}{0.75,1,0,0}
\newcommand{\doihref}[2]{\href{#1}{\color{magenta}{#2}}}
\newcommand{\adshref}[2]{\href{#1}{\color{mypurp}{#2}}}
\newcommand{\blackhref}[2]{\href{#1}{\color{black}{#2}}}

%% Define a format for the printtext commands in
%% DeclareBibliographyDriver to make links for the doi, ads link and
%% arxiv link:
\DeclareFieldFormat{doilink}{
  \iffieldundef{doi}{#1}{\doihref{http://dx.doi.org/\thefield{doi}}{#1}}}
\DeclareFieldFormat{adslink}{
    \iffieldundef{iswc}{#1}{\adshref{\thefield{iswc}}{#1}}}
\DeclareFieldFormat{arxivlink}{
  \iffieldundef{eprint}{#1}{\href{http://arxiv.org/abs/\thefield{eprint}}{#1}}}

\DeclareListFormat{doiforbook}{
  \iffieldundef{doi}{#1}{\doihref{http://dx.doi.org/\thefield{doi}}{#1}}}
\DeclareFieldFormat{googlebookslink}{
    \iffieldundef{iswc}{#1}{\adshref{\thefield{iswc}}{#1}}}

%% Set the formatting to make the last three values into the
%% appropriate link. Note that the % signs are necessary. Without
%% them, the items will be indented.
\DeclareBibliographyDriver{article}{%
  \usebibmacro{bibindex}%
  \usebibmacro{begentry}%
  \usebibmacro{author/translator+others}%
  \newunit%
  \ifdefined\makethesis\printtext{\usebibmacro{title}}\fi%
  \newunit%
  \printtext[doilink]{\usebibmacro{journal}}%
  \addcomma%
  \printtext[adslink]{\printfield{volume}}%
  \addcomma%
  \printtext[arxivlink]{\printfield{pages}}%
  \addperiod%
}

\DeclareBibliographyDriver{book}{%
  \usebibmacro{bibindex}%
  \usebibmacro{begentry}%
  \usebibmacro{author/translator+others}%
  \newunit%
  \printtext{\usebibmacro{title}}%
  \addperiod%
  \addspace%
  \printlist[doiforbook]{publisher}%
  \addcomma%
  \addspace%
  \printfield[googlebookslink]{edition}%
  \printtext{ ed.}%
  \addperiod%
}

%% In order to have et al. instead of et al.,:
\renewcommand*{\nameyeardelim}{\addspace}

%% PGFPlots settings
%% -----------------
%
%% PGFPLOTS is a package in (La)TeX for making plots internally. It fits
%% nicely with the purpose of a reproducible project. But it isn't
%% mandatory. Therefore if you don't need it, just comment/delete the line
%% that includes this file in the top LaTeX source ('paper.tex').
%
%% However, TiKZ and PGFPlots are the recommended way to include figures
%% and plots in your paper. There are two main reasons: 1) it follows the
%% same LaTeX settings as the text of the paper, so the figures will be in
%% the exact same settings (for example font or lines) as the main body of
%% the papers. 2) It doesn't require any extra dependency (it is
%% distributed as part of TeX-live). Adding specific programs/libraries for
%% plots can greatly increase the number of dependencies for the
%% project. For example Python's Matplotlib library is indeed very good,
%% but it requires Python and Numpy. The latter is not easy to build from
%% source, so after a few years, installing the required version can be
%% very frustrating.
%
%% Keeping all BibLaTeX settings in a separate preamble was done in the
%% spirit of modularity to 1) easily managable, 2) If a similar BibLaTeX
%% configuration is necessary in another LaTeX compilation, this file can
%% just be copied there and used.
%
%% PGFPlots uses the (La)TeX TiKZ package to build plots. So we will first
%% do the settings that are necessary in TiKZ, and then go onto the actual
%% PGFPlots package.
%%
%% USAGE:
%
%%  - All plots are made within a 'tikz' directory (that must already be
%%    present in the location LaTeX is run).
%
%%  - Use '\includetikz{XXXX}' to make/use the figure. If a 'makepdf' LaTeX
%%    macro is not defined, then \includetikz will assume a 'XXXX.pdf' file
%%    exists in 'tex/tikz' and simply import it. If 'makepdf' is defined,
%%    then TiKZ/PGFPlot will be called to (possibly) build the plot based
%%    on 'tex/XXXX.tex'. Note that if the contents of 'tex/src/XXXX.tex'
%%    hasn't changed since the last build. TiKZ/PGFPlots won't rebuild the
%%    plot.
%
%% Copyright (C) 2018-2022 Mohammad Akhlaghi <mohammad@akhlaghi.org>
%
%% This file is part of Maneage (https://maneage.org).
%
%% This file is free software: you can redistribute it and/or modify it
%% under the terms of the GNU General Public License as published by the
%% Free Software Foundation, either version 3 of the License, or (at your
%% option) any later version.
%
%% This file is distributed in the hope that it will be useful, but WITHOUT
%% ANY WARRANTY; without even the implied warranty of MERCHANTABILITY or
%% FITNESS FOR A PARTICULAR PURPOSE.  See the GNU General Public License
%% for more details.
%
%% You should have received a copy of the GNU General Public License along
%% with this file.  If not, see <http://www.gnu.org/licenses/>.




%% Very general TiKZ settings. In particular, to allow faster processing
%% (not having to re-build the plots on every run), we are using the
%% externalization feature of TiKZ. With this option, TiKZ will build every
%% figure independently in a special directory afterwards it will include
%% the built figure in the final file. This has many advantages: 1) if the
%% code for the plot hasn't changed, then the plot won't be re-made (can be
%% slow with detailed plots). 2) You can use the PDFs of the individual
%% plots for other purposes (for example to include in slides) cleanly.
\usepackage{tikz}
\usetikzlibrary{external}
\tikzexternalize
\tikzsetexternalprefix{tikz/}





%% The '\includetikz' can be used to either build the figures using
%% PGFPlots (when '\makepdf' is defined), or use an existing file (when
%% '\makepdf' isn't defined). When making the PDF, it will set the output
%% figure name to be the same as the 'tex/src/XXXX.tex' file that contains
%% the PGFPlots source of the figure. In this way, when using the PDF, it
%% will also have the same name, thus allowing the figures to easily change
%% their place relative to others: figure ordering won't be a problem. This
%% is a problem by default because if an explicit name isn't set at the
%% start, tikz will make images based on their order in the paper.
%
%% This function takes two arguments:
%%     1) The base-name of the LaTeX file with the 'tikzpicture'
%%        environment. As mentioned above, this will also be the name of
%%        the produced figure.
%%     2) The settings to use with 'includegraphics' when an already-built
%%        file should be used.
\newcommand{\includetikz}[2]{%
  \ifdefined\makepdf%
    \tikzsetnextfilename{#1}%
    \input{tex/src/#1.tex}%
  \else
    \includegraphics[#2]{tex/tikz/#1.pdf}
  \fi
}





%% Uncomment the following lines for EPS and PS images. Note that you still
%% have to use the 'pdflatex' executable and also add a '[dvips]' option to
%% graphicx.

%% \tikzset{external/system call={rm -f "\image".eps "\image".ps
%% "\image".dvi; latex \tikzexternalcheckshellescape -halt-on-error
%% -interaction=batchmode -jobname "\image" "\texsource";
%% dvips -o "\image".ps "\image".dvi;
%% ps2eps "\image.ps"}}





%% Inport and configure PGFPlots.
\usepackage{pgfplots}
\pgfplotsset{compat=newest}
\usepgfplotslibrary{groupplots}
\pgfplotsset{
  axis line style={thick},
  tick style={semithick},
  tick label style = {font=\footnotesize},
  every axis label = {font=\footnotesize},
  legend style = {font=\footnotesize},
  label style = {font=\footnotesize}
  }






%% Style of default paper (DELETE IF USING JOURNAL STYLES)
%% -------------------------------------------------------
%
%% This is primarily defined for the default Maneage paper style. So when
%% you later import your journal's style, delete this line (and these
%% comments). Also delete the file (to keep your project source branch
%% clean from files you don't need/use).
%% General paper's style settings.
%
%% This preamble can be completely ignored when including this TeX file in
%% another style. This is done because this LaTeX build is meant to be an
%% initial/internal phase or part of a larger effort, so it has a basic
%% style defined here as a preamble. To ignore it, uncomment or delete the
%% respective line in `paper.tex'.
%
%% Copyright (C) 2019-2021 Mohammad Akhlaghi <mohammad@akhlaghi.org>
%
%% This file is free software: you can redistribute it and/or modify it
%% under the terms of the GNU General Public License as published by the
%% Free Software Foundation, either version 3 of the License, or (at your
%% option) any later version.
%
%% This file is distributed in the hope that it will be useful, but WITHOUT
%% ANY WARRANTY; without even the implied warranty of MERCHANTABILITY or
%% FITNESS FOR A PARTICULAR PURPOSE.  See the GNU General Public License
%% for more details.
%
%% You should have received a copy of the GNU General Public License along
%% with this file.  If not, see <http://www.gnu.org/licenses/>.





%% Font.
\usepackage[T1]{fontenc}
\usepackage{newtxtext}
\usepackage{newtxmath}





%% Print size
\usepackage[a4paper, includeheadfoot, body={18.7cm, 24.5cm}]{geometry}





%% Set the distance between the columns if two columns:
\setlength{\columnsep}{0.75cm}





% To allow figures to take up more space on the top of the page:
\renewcommand{\topfraction}{.99}
\renewcommand{\bottomfraction}{.7}
\renewcommand{\textfraction}{.05}
\renewcommand{\floatpagefraction}{.99}
\renewcommand{\dbltopfraction}{.99}
\renewcommand{\dblfloatpagefraction}{.99}
\setcounter{topnumber}{1}
\setcounter{bottomnumber}{0}
\setcounter{totalnumber}{2}
\setcounter{dbltopnumber}{1}





%% Color related settings:
\usepackage{xcolor}
\color{black}                   % Text color
\definecolor{DarkBlue}{RGB}{0,0,90}






% figure and figure* ordering correction:
\usepackage{fixltx2e}





%% For editing the caption appearence. The `setspace' package defines
%% the `stretch' variable. `abovecaptionskip' is the distance between
%% the figure and the caption.
\usepackage{setspace, caption}
\captionsetup{font=footnotesize, labelfont={color=DarkBlue,bf}, skip=1pt}
\captionsetup[figure]{font={stretch=1, small}}
\setlength{\abovecaptionskip}{3pt plus 1pt minus 1pt}
\setlength{\belowcaptionskip}{-1.25em}






%% To make the footnotes align:
\usepackage[hang]{footmisc}
\setlength\footnotemargin{10pt}





%For including time in the title:
\usepackage{datetime}





%To make links to webpages and include document information in the
%properties of the PDF
\usepackage[
  colorlinks,
  urlcolor=blue,
  citecolor=blue,
  linkcolor=blue,
  linktocpage]{hyperref}
\renewcommand\UrlFont{\rmfamily}





%% Define the abstract environment
\renewenvironment{abstract}
 {\vspace{-0.5cm}\small%
  \list{}{%
    \setlength{\leftmargin}{2cm}%
    \setlength{\rightmargin}{\leftmargin}%
  }%
  \item\relax}
 {\endlist}





%% To keep the main page's code clean.
\newcommand{\includeabstract}[1]{%
\twocolumn[%
  \begin{@twocolumnfalse}%
    \maketitle%
    \begin{abstract}%
    #1%
    \end{abstract}%
    \vspace{1cm}%
  \end{@twocolumnfalse}%
  ]%
}





%% Basic header style
%% ------------------
%
%% The steps below are to use the necessary LaTeX packages to get the demo
%% Maneage paper running with a reasonably looking, custom paper style. If
%% you are using a custom journal style, feel free to delete these.

%% General page header settings.
\usepackage{fancyhdr}
\pagestyle{fancy}
\lhead{\footnotesize{\scshape Draft paper}, {\footnotesize nnn:i (pp), Year Month day}}
\rhead{\scshape\footnotesize YOUR-NAME et al.}
\cfoot{\thepage}
\setlength{\voffset}{0.75cm}
\setlength{\headsep}{0.2cm}
\setlength{\footskip}{0.75cm}
\renewcommand{\headrulewidth}{0pt}

%% Specific style for first page.
\fancypagestyle{firststyle}
{
  \lhead{\footnotesize{\scshape Draft paper}, nnn:i (pp), YYYY Month day\\
  \scriptsize \textcopyright YYYY, Your name. All rights reserved.}
  \rhead{\footnotesize \footnotesize \today, \currenttime\\}
}

%To set the style of the titles:
\usepackage{titlesec}
\titleformat{\section}
  {\centering\normalfont\uppercase}
  {\thesection.}
  {0em}
  { }
\titleformat{\subsection}
  {\centering\normalsize\slshape}
  {\thesubsection.}
  {0em}
  { }
\titleformat{\subsubsection}
  {\centering\small\slshape}
  {\thesubsubsection.}
  {0em}
  { }

% Basic Document information that goes into the PDF meta-data.
\hypersetup
{
    pdfauthor={YOUR NAME},
    pdfsubject={\projecttitle},
    pdftitle={\projecttitle},
    pdfkeywords={SOME, KEYWORDS, FOR, THE, PDF}
}

%% Title and author information
\usepackage{authblk}
\renewcommand\Authfont{\small\scshape}
\renewcommand\Affilfont{\footnotesize\normalfont}
\setlength{\affilsep}{0.2cm}

